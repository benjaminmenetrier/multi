\documentclass[francais]{beamer}
\usepackage[francais]{babel}
\usepackage[utf8]{inputenc}
\usepackage[T1]{fontenc}
\usepackage{lmodern}
\usepackage{amsmath, amssymb, amsfonts}




%CHOIX DU THEME et/ou DE SA COULEUR
% => essayer différents thèmes (en décommantant une des trois lignes suivantes)
%\usetheme{PaloAlto}
\usetheme{Madrid}

% => il est possible, pour un thème donné, de modifier seulement la couleur
\usecolortheme{default}
%\usecolortheme{whale}

%\useoutertheme[left]{sidebar}


%Pour le TITLEPAGE


\title[Nicolas Baillot d'Etivaux]{}


%Debut de la presentation :

\begin{document}


%Présentation:
\setbeamertemplate{blocks}[rounded]%
[shadow=false]


\begin{frame}{Content}
\begin{itemize}
\item Non linearity induced by H operator:\\
($Hx = (1-\alpha) x+ \alpha x^3$).
\item Number of outer loops vs. inner loops.\\
\item Non linearity induced by changing the resolution at outer loop level.
\item Varying the projective B matrix option.\\
\item Varying the interpolation method.
\end{itemize}
\end{frame}


% ------------------------------------------------------------------------------------------
% ------------------------------------------------------------------------------------------

\begin{frame}
\begin{center}
\huge{Non linearity induced by H operator:}\\
\vspace{+0.5cm}
\Large{Full resolution, varying $\alpha$ parameter with the same relinearization scheme: $no=4, ni=6$,\\
spectral interpolation and projective B matrix,
$\sigma^o=0.01$}
\end{center}
\end{frame}

%alpha=0
\begin{frame}{Full resolution; linear H; $J$ vs $J^{nl}$}
\begin{center}
\begin{figure}
\minipage{0.49\textwidth}
  \includegraphics[width=\linewidth]{img/Hnl_full_res/alpha/0/compare_j.png}
  \caption{$\alpha = 0$}
\endminipage\hfill
\minipage{0.49\textwidth}
  \includegraphics[width=\linewidth]{img/Hnl_full_res/alpha/0/compare_j_nl.png}
  \caption{$\alpha = 0$}
\endminipage
\end{figure}
\end{center}
\end{frame}

%alpha=0.01
\begin{frame}{Full resolution; non linear H; $J$ vs $J^{nl}$}
\begin{center}
\begin{figure}
\minipage{0.49\textwidth}
  \includegraphics[width=\linewidth]{img/Hnl_full_res/alpha/0.01/compare_j.png}
  \caption{$\alpha = 0.01$}
\endminipage\hfill
\minipage{0.49\textwidth}
  \includegraphics[width=\linewidth]{img/Hnl_full_res/alpha/0.01/compare_j_nl.png}
  \caption{$\alpha = 0.01$}
\endminipage
\end{figure}
\end{center}
\end{frame}

%alpha=0.02
\begin{frame}{Full resolution; non linear H; $J$ vs $J^{nl}$}
\begin{center}
\begin{figure}
\minipage{0.49\textwidth}
  \includegraphics[width=\linewidth]{img/Hnl_full_res/alpha/0.02/compare_j.png}
  \caption{$\alpha = 0.02$}
\endminipage\hfill
\minipage{0.49\textwidth}
  \includegraphics[width=\linewidth]{img/Hnl_full_res/alpha/0.02/compare_j_nl.png}
  \caption{$\alpha = 0.02$}
\endminipage
\end{figure}
\end{center}
\end{frame}

%alpha=0.05
\begin{frame}{Full resolution; non linear H; $J$ vs $J^{nl}$}
\begin{center}
\begin{figure}
\minipage{0.49\textwidth}
  \includegraphics[width=\linewidth]{img/Hnl_full_res/alpha/0.05/compare_j.png}
  \caption{$\alpha = 0.05$}
\endminipage\hfill
\minipage{0.49\textwidth}
  \includegraphics[width=\linewidth]{img/Hnl_full_res/alpha/0.05/compare_j_nl.png}
  \caption{$\alpha = 0.05$}
\endminipage
\end{figure}
\end{center}
\end{frame}

%alpha=0.1
\begin{frame}{Full resolution; non linear H; $J$ vs $J^{nl}$}
\begin{center}
\begin{figure}
\minipage{0.49\textwidth}
  \includegraphics[width=\linewidth]{img/Hnl_full_res/alpha/0.1/compare_j.png}
  \caption{$\alpha = 0.1$}
\endminipage\hfill
\minipage{0.49\textwidth}
  \includegraphics[width=\linewidth]{img/Hnl_full_res/alpha/0.1/compare_j_nl.png}
  \caption{$\alpha = 0.1$}
\endminipage
\end{figure}
\end{center}
\end{frame}

%alpha=0.5
\begin{frame}{Full resolution; non linear H; $J$ vs $J^{nl}$}
\begin{center}
\begin{figure}
\minipage{0.49\textwidth}
  \includegraphics[width=\linewidth]{img/Hnl_full_res/alpha/0.5/compare_j.png}
  \caption{$\alpha = 0.5$}
\endminipage\hfill
\minipage{0.49\textwidth}
  \includegraphics[width=\linewidth]{img/Hnl_full_res/alpha/0.5/compare_j_nl.png}
  \caption{$\alpha = 0.5$}
\endminipage
\end{figure}
\end{center}
\end{frame}

%alpha=1
\begin{frame}{Full resolution; non linear H; $J$ vs $J^{nl}$}
\begin{center}
\begin{figure}
\minipage{0.49\textwidth}
  \includegraphics[width=\linewidth]{img/Hnl_full_res/alpha/1/compare_j.png}
  \caption{$\alpha = 1$}
\endminipage\hfill
\minipage{0.49\textwidth}
  \includegraphics[width=\linewidth]{img/Hnl_full_res/alpha/1/compare_j_nl.png}
  \caption{$\alpha = 1$}
\endminipage
\end{figure}
\end{center}
\end{frame}


\begin{frame}{Conclusion on the non linearity induced by H}
 \begin{itemize}
  \item Very sensitive to $\alpha$ even for small values.
  \item The case with $\alpha = 1$ seems better than the case with $\alpha = 0.5$...\\
  $\longrightarrow$ What could be the reason for it ?
  \item It seems that there are too much inner loops before relinearization but the iteration at which the "jump" occurs seems NOT correlated to the value of $\alpha$.
 \end{itemize}
 \vspace{+0.5cm}
$\longrightarrow$ Need to study the number of inner iterations vs. outer iterations.
\end{frame}

% ------------------------------------------------------------------------------------------
% ------------------------------------------------------------------------------------------

\begin{frame}
\begin{center}
\large{Full resolution, varying the number of inner and outer loops with a non linear H and the same total number of iterations ($n_o \times n_i =24$)\\
(spectral interpolation and projective B matrix, $\sigma^o=0.01$)}
\end{center}
\end{frame}

% ------------------------------------------------------------------------------------------
%alpha = 0.05

%Jnl
\begin{frame}{Full resolution; non linear H ($\alpha = 0.05$): $J^{nl}$}
\begin{center}
\begin{figure}
\minipage{0.49\textwidth}
  \includegraphics[width=\linewidth]{img/Hnl_full_res/ni_vs_no/alpha0.05/no2_ni12/compare_j_nl.png}
  \caption{$n_o = 2, n_i = 12$}
\endminipage\hfill
\minipage{0.49\textwidth}
  \includegraphics[width=\linewidth]{img/Hnl_full_res/ni_vs_no/alpha0.05/no4_ni6/compare_j_nl.png}
  \caption{$n_o = 4, n_i = 6$}
\endminipage
\end{figure}
\end{center}
\end{frame}

\begin{frame}{Full resolution; non linear H ($\alpha = 0.05$): $J^{nl}$}
\begin{center}
\begin{figure}
\minipage{0.49\textwidth}
  \includegraphics[width=\linewidth]{img/Hnl_full_res/ni_vs_no/alpha0.05/no4_ni6/compare_j_nl.png}
  \caption{$n_o = 4, n_i = 6$}
\endminipage\hfill
\minipage{0.49\textwidth}
  \includegraphics[width=\linewidth]{img/Hnl_full_res/ni_vs_no/alpha0.05/no6_ni4/compare_j_nl.png}
  \caption{$n_o = 6, n_i = 4$}
\endminipage
\end{figure}
\end{center}
\end{frame}

%J
\begin{frame}{Full resolution; non linear H ($\alpha = 0.05$): $J$}
\begin{center}
\begin{figure}
\minipage{0.49\textwidth}
  \includegraphics[width=\linewidth]{img/Hnl_full_res/ni_vs_no/alpha0.05/no2_ni12/compare_j.png}
  \caption{$n_o = 2, n_i = 12$}
\endminipage\hfill
\minipage{0.49\textwidth}
  \includegraphics[width=\linewidth]{img/Hnl_full_res/ni_vs_no/alpha0.05/no4_ni6/compare_j.png}
  \caption{$n_o = 4, n_i = 6$}
\endminipage
\end{figure}
\end{center}
\end{frame}

\begin{frame}{Full resolution; non linear H ($\alpha = 0.05$): $J$}
\begin{center}
\begin{figure}
\minipage{0.49\textwidth}
  \includegraphics[width=\linewidth]{img/Hnl_full_res/ni_vs_no/alpha0.05/no4_ni6/compare_j.png}
  \caption{$n_o = 4, n_i = 6$}
\endminipage\hfill
\minipage{0.49\textwidth}
  \includegraphics[width=\linewidth]{img/Hnl_full_res/ni_vs_no/alpha0.05/no6_ni4/compare_j.png}
  \caption{$n_o = 6, n_i = 4$}
\endminipage
\end{figure}
\end{center}
\end{frame}

%------------------------------------------------------------------------------------------
%alpha = 0.1

%Jnl
\begin{frame}{Full resolution; non linear H ($\alpha = 0.1$): $J^{nl}$}
\begin{center}
\begin{figure}
\minipage{0.49\textwidth}
  \includegraphics[width=\linewidth]{img/Hnl_full_res/ni_vs_no/alpha0.1/no4_ni6/compare_j_nl.png}
  \caption{$n_o = 4, n_i = 6$}
\endminipage\hfill
\minipage{0.49\textwidth}
  \includegraphics[width=\linewidth]{img/Hnl_full_res/ni_vs_no/alpha0.1/no6_ni4/compare_j_nl.png}
  \caption{$n_o = 6, n_i = 4$}
\endminipage
\end{figure}
\end{center}
\end{frame}

%J
\begin{frame}{Full resolution; non linear H ($\alpha = 0.1$): $J$}
\begin{center}
\begin{figure}
\minipage{0.49\textwidth}
  \includegraphics[width=\linewidth]{img/Hnl_full_res/ni_vs_no/alpha0.1/no4_ni6/compare_j.png}
  \caption{$n_o = 4, n_i = 6$}
\endminipage\hfill
\minipage{0.49\textwidth}
  \includegraphics[width=\linewidth]{img/Hnl_full_res/ni_vs_no/alpha0.1/no6_ni4/compare_j.png}
  \caption{$n_o = 6, n_i = 4$}
\endminipage
\end{figure}
\end{center}
\end{frame}

%------------------------------------------------------------------------------------------
%alpha = 1

%Jnl
\begin{frame}{Full resolution; non linear H ($\alpha = 1$): $J^{nl}$}
\begin{center}
\begin{figure}
\minipage{0.49\textwidth}
  \includegraphics[width=\linewidth]{img/Hnl_full_res/ni_vs_no/alpha1/no4_ni6/compare_j_nl.png}
  \caption{$n_o = 4, n_i = 6$}
\endminipage\hfill
\minipage{0.49\textwidth}
  \includegraphics[width=\linewidth]{img/Hnl_full_res/ni_vs_no/alpha1/no6_ni4/compare_j_nl.png}
  \caption{$n_o = 6, n_i = 4$}
\endminipage
\end{figure}
\end{center}
\end{frame}

%J
\begin{frame}{Full resolution; non linear H ($\alpha = 1$): $J$}
\begin{center}
\begin{figure}
\minipage{0.49\textwidth}
  \includegraphics[width=\linewidth]{img/Hnl_full_res/ni_vs_no/alpha1/no4_ni6/compare_j.png}
  \caption{$n_o = 4, n_i = 6$}
\endminipage\hfill
\minipage{0.49\textwidth}
  \includegraphics[width=\linewidth]{img/Hnl_full_res/ni_vs_no/alpha1/no6_ni4/compare_j.png}
  \caption{$n_o = 6, n_i = 4$}
\endminipage
\end{figure}
\end{center}
\end{frame}

%------------------------------------------------------------------------------------------
\begin{frame}{Conclusion on the number of inner and outer loops}
\begin{itemize}
 \item As expected, the assimilation scheme with the more outer loops is equal or better than the others. 
 \item There is often a "jump" in the linear cost functions just after relinearization, BUT it seems not necessarily correlated to the behaviour of the non linear cost function (or it is not trivial).
 \item Problem: The first inner iterations in the first case with 12 inner iterations seems better than the case with 6 inner iterations whereas the case with 4 inner iterations seems better than the case with 6 inner iterations:\\
 $\longrightarrow$ The problem is too much dependant on the initial background and observation states that are randomly generated (?): there is a difference of $10^7 - 10^8$ in the cost function at the beggining in these cases! 
\end{itemize}
\end{frame}

%------------------------------------------------------------------------------------------
%------------------------------------------------------------------------------------------
\begin{frame}
\begin{center}
\huge{Non linearity induced by the change of resolution between the outer loops:}\\
\vspace{+0.5cm}
 \Large{(Linear H operator, spectral interpolation and projective B matrix, $\sigma^o=0.01$, $no=2, ni=12$)}
\end{center}
\end{frame}

%projT sp ------------------------------------------------------------------------------------------
\begin{frame}{Linear H, spectral interpolation and projective B matrix}
\begin{center}
\begin{figure}
\minipage{0.49\textwidth}
  \includegraphics[width=\linewidth]{img/jump_cost/alpha0/11-101/sp/T/compare_j_nl.png}
  \caption{resolutions: 11 > 101}
\endminipage\hfill
\minipage{0.49\textwidth}
  \includegraphics[width=\linewidth]{img/jump_cost/alpha0/51-101/sp/T/compare_j_nl.png}
  \caption{resolutions: 51 > 101}
\endminipage
\end{figure}
(The jump is also present in the $J^o$ and the residue)
\end{center}
\end{frame}


\begin{frame}{Linear H, spectral interpolation and projective B matrix}
\begin{center}
\begin{figure}
\minipage{0.49\textwidth}
  \includegraphics[width=\linewidth]{img/jump_cost/alpha0/51-101/sp/T/compare_j_nl.png}
  \caption{resolutions: 51 > 101}
\endminipage\hfill
\minipage{0.49\textwidth}
  \includegraphics[width=\linewidth]{img/jump_cost/alpha0/91-101/sp/T/compare_j_nl.png}
  \caption{resolutions: 91 > 101}
\endminipage
\end{figure}
\end{center}
\end{frame}

%jb
\begin{frame}{Linear H, spectral interpolation and projective B matrix}
\begin{center}
\begin{figure}
\minipage{0.49\textwidth}
  \includegraphics[width=\linewidth]{img/jump_cost/alpha0/11-101/sp/T/compare_jb_nl.png}
  \caption{resolutions: 11 > 101}
\endminipage\hfill
\minipage{0.49\textwidth}
  \includegraphics[width=\linewidth]{img/jump_cost/alpha0/51-101/sp/T/compare_jb_nl.png}
  \caption{resolutions: 51 > 101}
\endminipage
\end{figure}
\end{center}
\end{frame}


\begin{frame}{Linear H, spectral interpolation and projective B matrix}
\begin{center}
\begin{figure}
\minipage{0.49\textwidth}
  \includegraphics[width=\linewidth]{img/jump_cost/alpha0/51-101/sp/T/compare_jb_nl.png}
  \caption{resolutions: 51 > 101}
\endminipage\hfill
\minipage{0.49\textwidth}
  \includegraphics[width=\linewidth]{img/jump_cost/alpha0/91-101/sp/T/compare_jb_nl.png}
  \caption{resolutions: 91 > 101}
\endminipage
\end{figure}
\end{center}
\end{frame}

% diff methods %------------------------------------------------------------------------------------------
% diff methods %------------------------------------------------------------------------------------------

\begin{frame}
\begin{center}
\huge{Varying the projective B matrix option:}\\
\vspace{+0.5cm}
 \Large{(Linear H operator, spectral interpolation, $\sigma^o=0.01$, $no=2, ni=12$)}
\end{center}
\end{frame}

% projF sp %------------------------------------------------------------------------------------------
\begin{frame}{Linear H, spectral interpolation and NON projective B matrix}
\begin{center}
\begin{figure}
\minipage{0.49\textwidth}
  \includegraphics[width=\linewidth]{img/jump_cost/alpha0/11-101/sp/F/compare_j_nl.png}
  \caption{resolutions: 11 > 101}
\endminipage\hfill
\minipage{0.49\textwidth}
  \includegraphics[width=\linewidth]{img/jump_cost/alpha0/51-101/sp/F/compare_j_nl.png}
  \caption{resolutions: 51 > 101}
\endminipage
\end{figure}
\end{center}
\end{frame}

\begin{frame}{Linear H, spectral interpolation and NON projective B matrix}
\begin{center}
\begin{figure}
\minipage{0.49\textwidth}
  \includegraphics[width=\linewidth]{img/jump_cost/alpha0/51-101/sp/F/compare_j_nl.png}
  \caption{resolutions: 51 > 101}
\endminipage\hfill
\minipage{0.49\textwidth}
  \includegraphics[width=\linewidth]{img/jump_cost/alpha0/91-101/sp/F/compare_j_nl.png}
  \caption{resolutions: 91 > 101}
\endminipage
\end{figure}
\end{center}
\end{frame}

%jb
\begin{frame}{Linear H, spectral interpolation and NON projective B matrix}
\begin{center}
\begin{figure}
\minipage{0.49\textwidth}
  \includegraphics[width=\linewidth]{img/jump_cost/alpha0/11-101/sp/F/compare_jb_nl.png}
  \caption{resolutions: 11 > 101}
\endminipage\hfill
\minipage{0.49\textwidth}
  \includegraphics[width=\linewidth]{img/jump_cost/alpha0/51-101/sp/F/compare_jb_nl.png}
  \caption{resolutions: 51 > 101}
\endminipage
\end{figure}
\end{center}
\end{frame}

\begin{frame}{Linear H, spectral interpolation and NON projective B matrix}
\begin{center}
\begin{figure}
\minipage{0.49\textwidth}
  \includegraphics[width=\linewidth]{img/jump_cost/alpha0/51-101/sp/F/compare_jb_nl.png}
  \caption{resolutions: 51 > 101}
\endminipage\hfill
\minipage{0.49\textwidth}
  \includegraphics[width=\linewidth]{img/jump_cost/alpha0/91-101/sp/F/compare_jb_nl.png}
  \caption{resolutions: 91 > 101}
\endminipage
\end{figure}
\end{center}
\end{frame}

\begin{frame}{Short conclusions on the projective B matrix (1)}
\begin{center}
\begin{itemize}
 \item There might be a jump in the cost function at the first inner iteration of a new outer loop which is also present in the residue and in the $J^o$ and seems due to the change of resolution..\\   
 \item There is no differences between the methods if the B matrix is projective and if the interpolation is transitive (spectral).\\
 \item There are differences if the B matrix is not projective, and it occurs after the first outer loop.\\
 \item The higher the change of resolution, the higher the difference between the methods.\\
\end{itemize}
\end{center}
\end{frame}

\begin{frame}{Short conclusions on the projective B matrix (2)}
\begin{center}
\begin{itemize}
 \item In these cases, the alternative and standard methods give the same results and differ from the theoretical one when the B matrix is not projective.\\
 \item The theoretical method seems to give the same results for the $J^o$ with or without a projective B matrix while the $J^b$ is different (and shows a decrease after the first outer loop).
\end{itemize}
\end{center}
\end{frame}

% sigmabvar ------------------------------------------------------------------------------------------
\begin{frame}{Checking the effects of $\sigma^b_{var}$}
\begin{center}
\begin{figure}
\minipage{0.49\textwidth}
  \includegraphics[width=\linewidth]{img/sigmabvar/xt/0.1/xt.png}
  \caption{$\sigma^b_{var} = 0.1$}
\endminipage\hfill
\minipage{0.49\textwidth}
  \includegraphics[width=\linewidth]{img/sigmabvar/xt/1/xt.png}
  \caption{$\sigma^b_{var} = 1$}
\endminipage
\end{figure}
\end{center}
\end{frame}
%------------------------------------------------------------------------------------------



%------------------------------------------------------------------------------------------
%------------------------------------------------------------------------------------------


%------------------------------------------------------------------------------------------

% Conclusions ------------------------------------------------------------------------------------------
%------------------------------------------------------------------------------------------


\usebackgroundtemplate{}




\end{document}
