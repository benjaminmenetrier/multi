\documentclass[francais]{beamer}
\usepackage[francais]{babel}
\usepackage[utf8]{inputenc}
\usepackage[T1]{fontenc}
\usepackage{lmodern}
\usepackage{amsmath, amssymb, amsfonts}




%CHOIX DU THEME et/ou DE SA COULEUR
% => essayer différents thèmes (en décommantant une des trois lignes suivantes)
%\usetheme{PaloAlto}
\usetheme{Madrid}

% => il est possible, pour un thème donné, de modifier seulement la couleur
\usecolortheme{default}
%\usecolortheme{whale}

%\useoutertheme[left]{sidebar}


%Pour le TITLEPAGE


\title[Nicolas Baillot d'Etivaux]{}


%Debut de la presentation :

\begin{document}


%Présentation:
\setbeamertemplate{blocks}[rounded]%
[shadow=false]


\begin{frame}{Content}
\begin{itemize}
\item Influence of the $\sigma^o$ and $n_{obs}$.
\item Non linearity induced by H operator:\\
($Hx = (1-\alpha) x+ \alpha x^3$).
\item Non linearity induced by changing the resolution at outer loop level.
\end{itemize}
\end{frame}


\begin{frame}{Influence of the $\sigma^o$}
\begin{itemize}
\item Tuning the $\sigma^o$ only modifies the magnitude of $J^o$, which is anyway several orders of magnitudes higher than $J^b$.\\
\item Two results obtained with different $\sigma^o$ have identical behaviour regarding the cost functions, and the ratio between the two $J^o$ is equal to ${(\Delta \sigma^o)}^2$.
\item Thus, in the following, I decided to arbitrarily use a fixed value of $\sigma^o = 0.01$.
\item Tuning the number of observation also results in a change of magnitude but not in qualitative modifications of the cost functions.
\end{itemize}
\end{frame}

% ------------------------------------------------------------------------------------------
% ------------------------------------------------------------------------------------------

\begin{frame}
\begin{center}
\large{Full resolution, varying $\alpha$ parameter with the same relinearization scheme: $no=4, ni=6$,\\
spectral interpolation and projective B matrix,
$\sigma^o=0.01$}
\end{center}
\end{frame}

%alpha=0
\begin{frame}{Full resolution; linear H; $J$ vs $J^{nl}$}
\begin{center}
\begin{figure}
\minipage{0.49\textwidth}
  \includegraphics[width=\linewidth]{img/Hnl_full_res/alpha/0/compare_j.png}
  \caption{$\alpha = 0$}
\endminipage\hfill
\minipage{0.49\textwidth}
  \includegraphics[width=\linewidth]{img/Hnl_full_res/alpha/0/compare_j_nl.png}
  \caption{$\alpha = 0$}
\endminipage
\end{figure}
\end{center}
\end{frame}

%alpha=0.01
\begin{frame}{Full resolution; non linear H; $J$ vs $J^{nl}$}
\begin{center}
\begin{figure}
\minipage{0.49\textwidth}
  \includegraphics[width=\linewidth]{img/Hnl_full_res/alpha/0.01/compare_j.png}
  \caption{$\alpha = 0.01$}
\endminipage\hfill
\minipage{0.49\textwidth}
  \includegraphics[width=\linewidth]{img/Hnl_full_res/alpha/0.01/compare_j_nl.png}
  \caption{$\alpha = 0.01$}
\endminipage
\end{figure}
\end{center}
\end{frame}

%alpha=0.02
\begin{frame}{Full resolution; non linear H; $J$ vs $J^{nl}$}
\begin{center}
\begin{figure}
\minipage{0.49\textwidth}
  \includegraphics[width=\linewidth]{img/Hnl_full_res/alpha/0.02/compare_j.png}
  \caption{$\alpha = 0.02$}
\endminipage\hfill
\minipage{0.49\textwidth}
  \includegraphics[width=\linewidth]{img/Hnl_full_res/alpha/0.02/compare_j_nl.png}
  \caption{$\alpha = 0.02$}
\endminipage
\end{figure}
\end{center}
\end{frame}

%alpha=0.05
\begin{frame}{Full resolution; non linear H; $J$ vs $J^{nl}$}
\begin{center}
\begin{figure}
\minipage{0.49\textwidth}
  \includegraphics[width=\linewidth]{img/Hnl_full_res/alpha/0.05/compare_j.png}
  \caption{$\alpha = 0.05$}
\endminipage\hfill
\minipage{0.49\textwidth}
  \includegraphics[width=\linewidth]{img/Hnl_full_res/alpha/0.05/compare_j_nl.png}
  \caption{$\alpha = 0.05$}
\endminipage
\end{figure}
\end{center}
\end{frame}

%alpha=0.1
\begin{frame}{Full resolution; non linear H; $J$ vs $J^{nl}$}
\begin{center}
\begin{figure}
\minipage{0.49\textwidth}
  \includegraphics[width=\linewidth]{img/Hnl_full_res/alpha/0.1/compare_j.png}
  \caption{$\alpha = 0.1$}
\endminipage\hfill
\minipage{0.49\textwidth}
  \includegraphics[width=\linewidth]{img/Hnl_full_res/alpha/0.1/compare_j_nl.png}
  \caption{$\alpha = 0.1$}
\endminipage
\end{figure}
\end{center}
\end{frame}

%alpha=0.5
\begin{frame}{Full resolution; non linear H; $J$ vs $J^{nl}$}
\begin{center}
\begin{figure}
\minipage{0.49\textwidth}
  \includegraphics[width=\linewidth]{img/Hnl_full_res/alpha/0.5/compare_j.png}
  \caption{$\alpha = 0.5$}
\endminipage\hfill
\minipage{0.49\textwidth}
  \includegraphics[width=\linewidth]{img/Hnl_full_res/alpha/0.5/compare_j_nl.png}
  \caption{$\alpha = 0.5$}
\endminipage
\end{figure}
\end{center}
\end{frame}

%alpha=1
\begin{frame}{Full resolution; non linear H; $J$ vs $J^{nl}$}
\begin{center}
\begin{figure}
\minipage{0.49\textwidth}
  \includegraphics[width=\linewidth]{img/Hnl_full_res/alpha/1/compare_j.png}
  \caption{$\alpha = 1$}
\endminipage\hfill
\minipage{0.49\textwidth}
  \includegraphics[width=\linewidth]{img/Hnl_full_res/alpha/1/compare_j_nl.png}
  \caption{$\alpha = 1$}
\endminipage
\end{figure}
\end{center}
\end{frame}


\begin{frame}{Conclusion on the non linearity induced by H}
 \begin{itemize}
  \item Very sensitive to $\alpha$ even for small values.
  \item The case with $\alpha = 1$ seems better than the case with $\alpha = 0.05$...\\
  $\longrightarrow$ What could be the reason for it ?
  \item It seems that there are too much inner loops before relinearization but the iteration at which the "jump" occurs seems NOT correlated to the value of $\alpha$.
 \end{itemize}
 \vspace{+0.5cm}
$\longrightarrow$ Need to study the number of inner iterations vs. outer iterations.
\end{frame}

% ------------------------------------------------------------------------------------------
% ------------------------------------------------------------------------------------------

\begin{frame}
\begin{center}
\large{Full resolution, varying the number of inner and outer loops with a non linear H and the same total number of iterations ($n_o \times n_i =24$)\\
(spectral interpolation and projective B matrix,
$\sigma^o=0.01$)}
\end{center}
\end{frame}

% ------------------------------------------------------------------------------------------
%alpha = 0.05

%Jnl
\begin{frame}{Full resolution; non linear H ($\alpha = 0.05$): $J^{nl}$}
\begin{center}
\begin{figure}
\minipage{0.49\textwidth}
  \includegraphics[width=\linewidth]{img/Hnl_full_res/ni_vs_no/alpha0.05/no2_ni12/compare_j_nl.png}
  \caption{$n_o = 2, n_i = 12$}
\endminipage\hfill
\minipage{0.49\textwidth}
  \includegraphics[width=\linewidth]{img/Hnl_full_res/ni_vs_no/alpha0.05/no4_ni6/compare_j_nl.png}
  \caption{$n_o = 4, n_i = 6$}
\endminipage
\end{figure}
\end{center}
\end{frame}

\begin{frame}{Full resolution; non linear H ($\alpha = 0.05$): $J^{nl}$}
\begin{center}
\begin{figure}
\minipage{0.49\textwidth}
  \includegraphics[width=\linewidth]{img/Hnl_full_res/ni_vs_no/alpha0.05/no4_ni6/compare_j_nl.png}
  \caption{$n_o = 4, n_i = 6$}
\endminipage\hfill
\minipage{0.49\textwidth}
  \includegraphics[width=\linewidth]{img/Hnl_full_res/ni_vs_no/alpha0.05/no6_ni4/compare_j_nl.png}
  \caption{$n_o = 6, n_i = 4$}
\endminipage
\end{figure}
\end{center}
\end{frame}

%J
\begin{frame}{Full resolution; non linear H ($\alpha = 0.05$): $J$}
\begin{center}
\begin{figure}
\minipage{0.49\textwidth}
  \includegraphics[width=\linewidth]{img/Hnl_full_res/ni_vs_no/alpha0.05/no2_ni12/compare_j.png}
  \caption{$n_o = 2, n_i = 12$}
\endminipage\hfill
\minipage{0.49\textwidth}
  \includegraphics[width=\linewidth]{img/Hnl_full_res/ni_vs_no/alpha0.05/no4_ni6/compare_j.png}
  \caption{$n_o = 4, n_i = 6$}
\endminipage
\end{figure}
\end{center}
\end{frame}

\begin{frame}{Full resolution; non linear H ($\alpha = 0.05$): $J$}
\begin{center}
\begin{figure}
\minipage{0.49\textwidth}
  \includegraphics[width=\linewidth]{img/Hnl_full_res/ni_vs_no/alpha0.05/no4_ni6/compare_j.png}
  \caption{$n_o = 4, n_i = 6$}
\endminipage\hfill
\minipage{0.49\textwidth}
  \includegraphics[width=\linewidth]{img/Hnl_full_res/ni_vs_no/alpha0.05/no6_ni4/compare_j.png}
  \caption{$n_o = 6, n_i = 4$}
\endminipage
\end{figure}
\end{center}
\end{frame}

%------------------------------------------------------------------------------------------
%alpha = 0.1

%Jnl
\begin{frame}{Full resolution; non linear H ($\alpha = 0.1$): $J^{nl}$}
\begin{center}
\begin{figure}
\minipage{0.49\textwidth}
  \includegraphics[width=\linewidth]{img/Hnl_full_res/ni_vs_no/alpha0.1/no4_ni6/compare_j_nl.png}
  \caption{$n_o = 4, n_i = 6$}
\endminipage\hfill
\minipage{0.49\textwidth}
  \includegraphics[width=\linewidth]{img/Hnl_full_res/ni_vs_no/alpha0.1/no6_ni4/compare_j_nl.png}
  \caption{$n_o = 6, n_i = 4$}
\endminipage
\end{figure}
\end{center}
\end{frame}

%J
\begin{frame}{Full resolution; non linear H ($\alpha = 0.1$): $J$}
\begin{center}
\begin{figure}
\minipage{0.49\textwidth}
  \includegraphics[width=\linewidth]{img/Hnl_full_res/ni_vs_no/alpha0.1/no4_ni6/compare_j.png}
  \caption{$n_o = 4, n_i = 6$}
\endminipage\hfill
\minipage{0.49\textwidth}
  \includegraphics[width=\linewidth]{img/Hnl_full_res/ni_vs_no/alpha0.1/no6_ni4/compare_j.png}
  \caption{$n_o = 6, n_i = 4$}
\endminipage
\end{figure}
\end{center}
\end{frame}

%------------------------------------------------------------------------------------------
%alpha = 1

%Jnl
\begin{frame}{Full resolution; non linear H ($\alpha = 1$): $J^{nl}$}
\begin{center}
\begin{figure}
\minipage{0.49\textwidth}
  \includegraphics[width=\linewidth]{img/Hnl_full_res/ni_vs_no/alpha1/no4_ni6/compare_j_nl.png}
  \caption{$n_o = 4, n_i = 6$}
\endminipage\hfill
\minipage{0.49\textwidth}
  \includegraphics[width=\linewidth]{img/Hnl_full_res/ni_vs_no/alpha1/no6_ni4/compare_j_nl.png}
  \caption{$n_o = 6, n_i = 4$}
\endminipage
\end{figure}
\end{center}
\end{frame}

%J
\begin{frame}{Full resolution; non linear H ($\alpha = 1$): $J$}
\begin{center}
\begin{figure}
\minipage{0.49\textwidth}
  \includegraphics[width=\linewidth]{img/Hnl_full_res/ni_vs_no/alpha1/no4_ni6/compare_j.png}
  \caption{$n_o = 4, n_i = 6$}
\endminipage\hfill
\minipage{0.49\textwidth}
  \includegraphics[width=\linewidth]{img/Hnl_full_res/ni_vs_no/alpha1/no6_ni4/compare_j.png}
  \caption{$n_o = 6, n_i = 4$}
\endminipage
\end{figure}
\end{center}
\end{frame}

%------------------------------------------------------------------------------------------
\begin{frame}{Conclusion on the number of inner and outer loops}
\begin{itemize}
 \item As expected, the assimilation scheme with the more outer loops is equal or better than the others. 
 \item There is often a "jump" in the linear cost functions just after relinearization, BUT it seems not necessarily correlated to the behaviour of the non linear cost function (or it is not trivial at all).
 \item Problem: The firsts inner iterations in the first case with 12 inner iterations seems better than the case with 6 inner iterations whereas the case with 6 inner iterations seems better than the case with 4 inner iterations:\\
 $\longrightarrow$ The problem is too much dependant on the initial background and observation states that are randomly generated (?): there is a difference of $10^7$ at the beggining in this case!
\end{itemize}
\end{frame}

%------------------------------------------------------------------------------------------
\begin{frame}
\begin{center}
 \large{Comparing the interpolation methods and projective B matrix options when changing the resolution between the outer loops}\\
 Reassuring: (No differences have been found between the interpolation methods, nor between the projective B matrix options when no resolution changes are made).
\end{center}
\end{frame}

\begin{frame}
\begin{center}
 \large{Non linearity induced by changing the resolution}
\end{center}
\end{frame}






%------------------------------------------------------------------------------------------
% Conclusions ------------------------------------------------------------------------------------------
\begin{frame}{Short conclusion and questions:}
\begin{itemize}
\item lala
\end{itemize}
\end{frame}
%------------------------------------------------------------------------------------------


\usebackgroundtemplate{}




\end{document}
