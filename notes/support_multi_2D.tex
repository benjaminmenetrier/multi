\documentclass[francais]{beamer}
\usepackage[francais]{babel}
\usepackage[utf8]{inputenc}
\usepackage[T1]{fontenc}
\usepackage{lmodern}
\usepackage{amsmath, amssymb, amsfonts}




%CHOIX DU THEME et/ou DE SA COULEUR
% => essayer différents thèmes (en décommantant une des trois lignes suivantes)
%\usetheme{PaloAlto}
\usetheme{Madrid}

% => il est possible, pour un thème donné, de modifier seulement la couleur
\usecolortheme{default}
%\usecolortheme{whale}

%\useoutertheme[left]{sidebar}


%Pour le TITLEPAGE


\title[Nicolas Baillot d'Etivaux]{Constraining the equation of state for dense matter through thermal emission of neutron stars}


%Debut de la presentation :

\begin{document}


%Présentation:
\setbeamertemplate{blocks}[rounded]%
[shadow=false]


\begin{frame}{Full resolution, no preconditionning: background state}
\begin{center}
$nx\times ny=101 \times 101$, $sigmabvar=0.1$, $L_b=0.1$.\\
\begin{figure}
 \includegraphics[scale=0.5]{./images/xb.png}
 \caption{$x^b$}
\end{figure} 
\end{center}
\end{frame}


\begin{frame}{Full resolution, no preconditionning: observations}
\begin{center}
$nx\times ny=101 \time 101$, $\sigma^o$=0.01, $n_{obs}=2000$.\\
\begin{figure}
 \includegraphics[scale=0.5]{./images/obs_coord.png}
 \caption{Observations}
\end{figure} 
\end{center}
\end{frame}

\begin{frame}{Full resolution, no preconditionning: $\sigma^b$ example}
\begin{center}
$nx\times ny=101 \times 101$, $sigmabvar=0.1$, $L_b=0.1$.\\
\begin{figure}
 \includegraphics[scale=0.5]{./images/sigmab.png}
 \caption{$\sigma^b$ field example}
\end{figure} 
\end{center}
\end{frame}

\begin{frame}{Full resolution, no preconditionning: dirac corr example}
\begin{center}
$nx\times ny=101 \times 101$, $sigmabvar=0.1$, $L_b=0.1$.\\
\begin{figure}
 \includegraphics[scale=0.5]{./images/dirac_cor.png}
 \caption{dirac corr example}
\end{figure} 
\end{center}
\end{frame}



% guess -------------------------------------------------------------
\begin{frame}{Full resolution, no preconditionning: guess}
\begin{center}
With lanczos algorithm:\\
(maybe there is a problem with the guess initialisation (to $x^b$ ?)):
\begin{figure}
\minipage{0.49\textwidth}
  \includegraphics[width=\linewidth]{images/lxg1.png}
  \caption{guess (outer iteration 1)}
\endminipage\hfill
\minipage{0.49\textwidth}
  \includegraphics[width=\linewidth]{images/lxg2.png}
  \caption{guess (outer iteration 2)}
\endminipage
\end{figure}
\end{center}
\end{frame}

\begin{frame}{Full resolution, no preconditionning: guess}
\begin{center}
With lanczos algorithm:
\begin{figure}
\minipage{0.49\textwidth}
  \includegraphics[width=\linewidth]{images/lxg3.png}
  \caption{guess (outer iteration 3)}
\endminipage \hfill
\minipage{0.49\textwidth}%
  \includegraphics[width=\linewidth]{images/lxg4.png}
  \caption{guess (outer iteration 4)}
\endminipage
\end{figure}
\end{center}
\end{frame}

\begin{frame}{Full resolution, no preconditionning: guess}
\begin{center}
With PlanczosIF algorithm:\\
(idem for the first guess?):
\begin{figure}
\minipage{0.49\textwidth}
  \includegraphics[width=\linewidth]{images/pxg1.png}
  \caption{guess (outer iteration 1)}
\endminipage\hfill
\minipage{0.49\textwidth}
  \includegraphics[width=\linewidth]{images/pxg2.png}
  \caption{guess (outer iteration 2)}
\endminipage
\end{figure}
\end{center}
\end{frame}

\begin{frame}{Full resolution, no preconditionning: guess}
\begin{center}
With PlanczosIF algorithm:
\begin{figure}
\minipage{0.49\textwidth}
  \includegraphics[width=\linewidth]{images/pxg3.png}
  \caption{guess (outer iteration 3)}
\endminipage \hfill
\minipage{0.49\textwidth}%
  \includegraphics[width=\linewidth]{images/pxg4.png}
  \caption{guess (outer iteration 4)}
\endminipage
\end{figure}
\end{center}
\end{frame}
%------------------------------

% hxg -------------------------------------------------------------
\begin{frame}{Full resolution, no preconditionning: $H x^g$}
\begin{center}
This example is obtained in model space but the results are strictly equal in control space (but they should not since the first guesses are different ?):
\begin{figure}
\minipage{0.49\textwidth}
  \includegraphics[width=\linewidth]{images/hxg1.png}
  \caption{$H x^g$ (outer iteration 1)}
\endminipage\hfill
\minipage{0.49\textwidth}
  \includegraphics[width=\linewidth]{images/hxg2.png}
  \caption{$H x^g$ (outer iteration 2)}
\endminipage
\end{figure}
\end{center}
\end{frame}

\begin{frame}{Full resolution, no preconditionning: $H x^g$}
\begin{center}
Reassuring: it seems to converge to small values of the $H x^g$
\begin{figure}
\minipage{0.49\textwidth}
  \includegraphics[width=\linewidth]{images/hxg3.png}
  \caption{$H x^g$ (outer iteration 3)}
\endminipage \hfill
\minipage{0.49\textwidth}%
  \includegraphics[width=\linewidth]{images/hxg4.png}
  \caption{$H x^g$ (outer iteration 4)}
\endminipage
\end{figure}
\end{center}
\end{frame}
%-------------------------------------------------------------

% innovation -------------------------------------------------------------
\begin{frame}{Full resolution, no preconditionning: innovations}
\begin{center}
This example is obtained in model space but the results are strictly equal in control space (but they should not since the first guesses are different ?):
\begin{figure}
\minipage{0.49\textwidth}
  \includegraphics[width=\linewidth]{images/d1.png}
  \caption{innovation (outer iteration 1)}
\endminipage\hfill
\minipage{0.49\textwidth}
  \includegraphics[width=\linewidth]{images/d2.png}
  \caption{innovation (outer iteration 2)}
\endminipage
\end{figure}
\end{center}
\end{frame}

\begin{frame}{Full resolution, no preconditionning: innovations}
\begin{center}
Reassuring: it seems to converge to small values of the innovation
\begin{figure}
\minipage{0.49\textwidth}
  \includegraphics[width=\linewidth]{images/d3.png}
  \caption{innovation (outer iteration 3)}
\endminipage \hfill
\minipage{0.49\textwidth}%
  \includegraphics[width=\linewidth]{images/d4.png}
  \caption{innovation (outer iteration 4)}
\endminipage
\end{figure}
\end{center}
\end{frame}
%-------------------------------------------------------------



% cost function:----------------------------------------------------------------------------
\begin{frame}{Full resolution, no preconditionning: Cost function}
\begin{center}
$\sigma^o$=0.01, $nx\times ny=101 \times 101$, $sigmabvar=0.1$, $n_{obs}=2000$, $L_b=0.1$\\
 \includegraphics[scale=0.3]{./images/jnl.png}
\end{center}
\end{frame}

\begin{frame}{Full resolution, no preconditionning: Cost function}
\begin{center}
$\sigma^o$=0.01, $nx\times ny=101 \times 101$, $sigmabvar=0.1$, $n_{obs}=2000$, $L_b=0.1$\\
 \includegraphics[scale=0.3]{./images/jonl.png}
\end{center}
\end{frame}

\begin{frame}{Full resolution, no preconditionning: Cost function}
\begin{center}
$\sigma^o$=0.01, $nx\times ny=101 \times 101$, $sigmabvar=0.1$, $n_{obs}=2000$, $L_b=0.1$\\
 \includegraphics[scale=0.3]{./images/jbnl.png}
\end{center}
\end{frame}

\begin{frame}{Full resolution, no preconditionning: Cost function}
\begin{center}
$\sigma^o$=0.01, $nx\times ny=101 \times 101$, $sigmabvar=0.1$, $n_{obs}=2000$, $L_b=0.1$\\
 \includegraphics[scale=0.3]{./images/jb.png}
\end{center}
\end{frame}

\begin{frame}{Full resolution, no preconditionning: Cost function}
\begin{center}
$\sigma^o$=0.01, $nx\times ny=101 \times 101$, $sigmabvar=0.1$, $n_{obs}=2000$, $L_b=0.1$\\
 \includegraphics[scale=0.3]{./images/beta.png}
\end{center}
\end{frame}

\begin{frame}{Full resolution, no preconditionning: Cost function}
\begin{center}
$\sigma^o$=0.01, $nx\times ny=101 \times 101$, $sigmabvar=0.1$, $n_{obs}=2000$, $L_b=0.1$\\
 \includegraphics[scale=0.3]{./images/rho.png}
\end{center}
\end{frame}


% Changing the resolution ------------------------------------------------------------------------------------------


\begin{frame}
\begin{center}
  \huge{Canging the resolution}
\end{center}
\end{frame}

% xb -------------------------------------------------------------
\begin{frame}{Changing resolution: $x^b$}
\begin{center}
\begin{figure}
\minipage{0.49\textwidth}
  \includegraphics[width=\linewidth]{images/xb1r.png}
  \caption{$x^b$ (outer iteration 1)}
\endminipage\hfill
\minipage{0.49\textwidth}
  \includegraphics[width=\linewidth]{images/xb2r.png}
  \caption{$x^b$ (outer iteration 2)}
\endminipage
\end{figure}
\end{center}
\end{frame}

\begin{frame}{Changing resolution: $x^b$}
\begin{center}
\begin{figure}
\minipage{0.49\textwidth}
  \includegraphics[width=\linewidth]{images/xb3r.png}
  \caption{$x^b$ (outer iteration 3)}
\endminipage \hfill
\minipage{0.49\textwidth}%
  \includegraphics[width=\linewidth]{images/xb4r.png}
  \caption{$x^b$ (outer iteration 4)}
\endminipage
\end{figure}
\end{center}
\end{frame}
%-------------------------------------------------------------


% guess -------------------------------------------------------------
\begin{frame}{Changing resolution, no preconditionning: guess}
\begin{center}
With lanczos algorithm:
\begin{figure}
\minipage{0.49\textwidth}
  \includegraphics[width=\linewidth]{images/lxg1r.png}
  \caption{guess (outer iteration 1)}
\endminipage\hfill
\minipage{0.49\textwidth}
  \includegraphics[width=\linewidth]{images/lxg2r.png}
  \caption{guess (outer iteration 2)}
\endminipage
\end{figure}
\end{center}
\end{frame}

\begin{frame}{Changing resolution, no preconditionning: guess}
\begin{center}
With lanczos algorithm:
\begin{figure}
\minipage{0.49\textwidth}
  \includegraphics[width=\linewidth]{images/lxg3r.png}
  \caption{guess (outer iteration 3)}
\endminipage \hfill
\minipage{0.49\textwidth}%
  \includegraphics[width=\linewidth]{images/lxg4r.png}
  \caption{guess (outer iteration 4)}
\endminipage
\end{figure}
\end{center}
\end{frame}

\begin{frame}{Changing resolution, no preconditionning: guess}
\begin{center}
With PlanczosIF algorithm:
\begin{figure}
\minipage{0.49\textwidth}
  \includegraphics[width=\linewidth]{images/pxg1r.png}
  \caption{guess (outer iteration 1)}
\endminipage\hfill
\minipage{0.49\textwidth}
  \includegraphics[width=\linewidth]{images/pxg2r.png}
  \caption{guess (outer iteration 2)}
\endminipage
\end{figure}
\end{center}
\end{frame}

\begin{frame}{Changing resolution, no preconditionning: guess}
\begin{center}
With PlanczosIF algorithm:
\begin{figure}
\minipage{0.49\textwidth}
  \includegraphics[width=\linewidth]{images/pxg3r.png}
  \caption{guess (outer iteration 3)}
\endminipage \hfill
\minipage{0.49\textwidth}%
  \includegraphics[width=\linewidth]{images/pxg4r.png}
  \caption{guess (outer iteration 4)}
\endminipage
\end{figure}
\end{center}
\end{frame}

\begin{frame}{Full resolution, no preconditionning: Cost function}
\begin{center}
 \includegraphics[scale=0.3]{./images/jnlr.png}
\end{center}
\end{frame}

%------------------------------------------------------------------------------------------


% Conclusions ------------------------------------------------------------------------------------------
\begin{frame}{Small conclusion:}
\begin{itemize}
 \item The code seems to converge to a correct solution according to the innovation, cost function and residual norm, both in full resolution or changing it.
 \item There is no significant difference between the three methods in both cases.
 \item There is a small difference between non-linear and linearized cost functions.
 \item There might be a problem with the guess (?).
 \item There is a problem with the inner increments (always at filling values except for the first one).
\end{itemize}
\end{frame}

\begin{frame}{To Do}
\begin{itemize}
 \item Save the obs values.
 \item Save the outer increments.
 \item Check the initialisation of the guess.
 \item Check the values of the inner increments.
 \item Question: why the dimensions of the 2 fields in netcdf file are (ny,nx) and not (nx,ny) ?
\end{itemize}
\end{frame}
%------------------------------------------------------------------------------------------






\usebackgroundtemplate{}




\end{document}
