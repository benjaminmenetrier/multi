\documentclass[francais]{beamer}
\usepackage[francais]{babel}
\usepackage[utf8]{inputenc}
\usepackage[T1]{fontenc}
\usepackage{lmodern}
\usepackage{amsmath, amssymb, amsfonts}




%CHOIX DU THEME et/ou DE SA COULEUR
% => essayer différents thèmes (en décommantant une des trois lignes suivantes)
%\usetheme{PaloAlto}
\usetheme{Madrid}

% => il est possible, pour un thème donné, de modifier seulement la couleur
\usecolortheme{default}
%\usecolortheme{whale}

%\useoutertheme[left]{sidebar}


%Pour le TITLEPAGE


\title[Nicolas Baillot d'Etivaux]{Constraining the equation of state for dense matter through thermal emission of neutron stars}


%Debut de la presentation :

\begin{document}


\usebackgroundtemplate{\includegraphics[scale = 0.6]{./images/merci.jpeg}}

%Titre :

\null

\vspace{-0.5cm}
\begin{frame}
	\hbox to \textwidth{
		\includegraphics[scale=0.10]{./images/ipnl.jpeg}\hfill
		\includegraphics[scale=0.08]{./images/logo_UdL.png}
		}
	
%\vspace{+1cm}
\begin{center}
\textcolor{white}{\Large{Constraining the equation of state for dense matter through thermal emission of neutron stars}}\\[+0.5cm]
\end{center}


\vspace{+1cm}
\textcolor{white}{Nicolas Baillot d'Etivaux - IPN Lyon}\\
\textcolor{white}{QGP-France 2016, Etretat}\\[+0.5cm]
\textcolor{white}{in collaboration with:} \\
\textcolor{white}{Jérôme Margueron - IPN Lyon } \\
\textcolor{white}{Hubert Hansen - IPN Lyon } \\		
\textcolor{white}{Natalie Webb - IRAP} \\
\textcolor{white}{Sébastien Guillot - Pontificia Universidad Católica de Chile} \\
\end{frame}

\usebackgroundtemplate{}





%Plan :






%Présentation:
\setbeamertemplate{blocks}[rounded]%
[shadow=false]

\color{white}



\section{Intro:}

\usebackgroundtemplate{\includegraphics[scale=0.8]{./images/backdark.jpg}}

\begin{frame}{The best laboratory for very high densities:}
	\begin{itemize}
	\color{white}
	\item Hadronic matter at very high density / Assymetric matter
	\item \textbf{Phase transition?}
\vspace{+0.3cm}	
\begin{center}
\includegraphics[scale=0.444]{./images/phase.jpeg}
\end{center}
\end{itemize}
\end{frame}



\begin{frame}{Hadronic matter, anyway!}
\textbf{Comparision between two "isotopes":}\\
\includegraphics[scale=1]{./images/nucleus3.jpg}\hfill
\includegraphics[scale=0.35]{./images/crabstar.jpg}\\
\vspace{+0.5cm}
$\simeq 10^{-15}$ m \hspace{+3cm} $\longrightarrow$ \hspace{+3cm} $\simeq 10^{3} $ m
\end{frame}

\begin{frame}{Neutron stars structure: }
\begin{itemize}
\color{white}
	\item Very high densities ($1.4M_{\odot},R \sim 10 km \rightarrow n \sim 10^{14} g.cm^{-3}$).
	\item tottally degenarated fermion gaz (nucleons? hyperons? mesons? quarks? ...)
\end{itemize}
\begin{center}
\includegraphics[scale=0.26]{./images/int_neutron}
\end{center}
\end{frame}



\begin{frame}{Link between EoS and M-R relation:}
T.O.V equations (Tolmann-Oppenheimer-Volkov):
{\scriptsize
\begin{equation}
\dfrac{dm(r)}{dr} =  4\pi r^2 \epsilon(r),
\end{equation}
\begin{equation}
\dfrac{dP(r)}{dr} =  -\dfrac{G\epsilon(r) m(r)}{r^2}\left(1+\dfrac{P(r)}{\epsilon(r)c^2}\right) \left( 1+\dfrac{4\pi P(r) r^3}{m(r)c^2}\right) \times  \left( 1- \dfrac{r_{sh}}{r} \dfrac{m(r)}{M}\right)^{-1}.
\end{equation}
}
Equation of state: $P(\epsilon)\leftrightarrow$ mass-radius relation.
\vspace{-0.2cm}
\begin{center}
\includegraphics[scale=0.35]{./images/eos}
\end{center}
\vspace{-0.4cm}	
{\scriptsize Anna Watts et al, 2014}
\end{frame}


\begin{frame}{Various equations of state vs. observations:}
\begin{center}
\includegraphics[scale=0.27]{images/demorest.jpg}\\

{\small (Demorest et. al. 2010)}
\end{center}
\end{frame}

\begin{frame}{Masses distribution of NS:}
\begin{center}
\includegraphics[scale=0.8]{./images/mass.jpeg}
\end{center}
\end{frame}

\begin{frame}{Radii distribution of NS:}
\begin{center}
$R\in [0;+\infty]$ km.
\end{center}
\end{frame}

\begin{frame}{Radii distribution of NS:}
\begin{center}
More seriously,\\
$R\in [5;30]$ km.
\end{center}
\end{frame}


\begin{frame}{Constraining nuclear equation of state from thermal radiation of neutron stars: }
\begin{itemize}
\color{white}
	\item Devellopement of pulsar X-ray astronomy $\rightarrow$ strong limits on the equation of 				state at high density (\"Ozel et al 2010 , Steiner et al 2010).
	\vspace{+0.3cm}
	\item Tool: Thermal radiation from the surface of neutron stars.
	\vspace{+0.3cm}
	\item Devellopement of atmosphere models (Heinke et al 2006).
	\vspace{+0.3cm}
	\item Applying Bayesian analysis for several qLMXBs (Guillot et al 2013, Lattimer \& Steiner 2014, 				Heinke et al 2014, \"Ozel et al 2015, Bogdanov 2016).
\end{itemize}
\end{frame}

\begin{frame}{The equation of state (EoS) for neutron stars interior:}
Various kinds of EoS:
\vspace{+0.5cm}
\begin{itemize}
\color{white}
	\item Purely nucleonic $\rightarrow$ no phase transition, \textbf{smooth EoS} with $n , y_e,T$.
	\item Phase transition:
		\begin{itemize}
		\color{white}
			\item hadronic matter: onset of hyperons, pion condensate ...
			\item pure quark stars (absolutely stable strange matter) ? 
			\item hybrid stars (QGP/color superconductivity in the core) ?
		\end{itemize}			
	\vspace{+0.5cm}
\end{itemize}
Motivation for choosing \textbf{pure nucleonic matter}:
\begin{itemize}
\color{white}
	\vspace{+0.5cm}
	\item smooth EoS
	\item extrapolation of nuclear physics knowledge (and uncertainties) towards high densities and low $Y_p$.
	\item \textbf{ Define M-R boundaries for smooth EoS.}
\end{itemize}
\end{frame}



\section{EoS:}






\begin{frame}{The EoS model based on empirical parameters:} 
	\begin{block}{Definition of the empirical parameters:}
	\begin{equation}
\epsilon(n,\delta)=\epsilon_{IS}+\delta^2 \epsilon_{IV}
	\end{equation}
	\begin{equation}
	\epsilon_{IS} = E_{sat}+\dfrac{1}{2}K_{sat}x^2 + \dfrac{1}{3!}Q_{sat}x^3+\dfrac{1}{4!}Z_{sat}x^4+o(x^5)
	\end{equation}
	\begin{equation}
	\epsilon_{IV} = E_{sym}+L_{sym}x + \dfrac{1}{2}K_{sym}x^2 + \dfrac{1}{3!}Q_{sym}x^3+\dfrac{1}{4!}Z_{sym}x^4+o(x^5)
	\end{equation}
	\end{block}
	
	\begin{block}{Taylor expansion around $n_0$ for the energy density:}
	\vspace{-0.3cm}
			$$\epsilon(n,\delta)=t(n,											\delta)+\sum^{N}_{\alpha\geq0}v_{\alpha}								(\delta)\frac{x(n)^{\alpha}}{\alpha!}u^N_{\alpha}(x) ,
			 \quad \delta  = \dfrac{(n_n-n_p)}{n},\quad x=\dfrac{(n-n_0)}{3n_0}$$
$v_\alpha(\delta) = v_{\alpha,IS} + v_{\alpha,IV}\delta^2$\\
{\scriptsize (Margueron, Casali, Gulminelli, in preparation)}
	\end{block}
\end{frame}	
	

\begin{frame}{Ability of the EoS to mimic known EoS: example of SLy5}
\begin{center}
\includegraphics[scale=0.55]{./images/sly51.jpg}
\end{center}
{\scriptsize (Margueron, Casali, Gulminelli, in preparation)}
\end{frame}

\begin{frame}{Ability of the EoS to mimic known EoS: example of SLy5}
\begin{center}
\includegraphics[scale=0.65]{./images/sly52.jpg}
\end{center}
\vspace{-0.5cm}
{\scriptsize (Margueron, Casali, Gulminelli, in preparation)}
\end{frame}

\begin{frame}{Our present knowledge of the empirical parameters:}
\begin{center}
Empirical parameters for various effective approaches:\\
\includegraphics[scale=0.4]{./images/tabpar.pdf}
\end{center}
\vspace{+0.3cm}
{\scriptsize (Margueron, Casali, Gulminelli, in preparation)}
\end{frame}


\usebackgroundtemplate{}
\begin{frame}{Dependence of M-R on the empirical parameters:}
\textcolor{black}{+ constrains: $0<v_s^2<c^2$ and $\epsilon_{IV}(n)>0$ for $n<4n_{sat}$ }\\
\includegraphics[scale=0.9]{./images/plot_b_tot2.pdf}\\
{\scriptsize \textcolor{black}{(Margueron, Casali, Gulminelli, in preparation)}}
\end{frame}


\usebackgroundtemplate{\includegraphics[scale=0.8]{./images/backdark.jpg}}

\begin{frame}{Dependence of M-R on the empirical parameters:}
\includegraphics[scale=0.4]{./images/eeos_1.pdf}\\
{\scriptsize (Margueron, Casali, Gulminelli, in preparation)}
\end{frame}



\begin{frame}{Dependence of M-R on the empirical parameters:}
\includegraphics[scale=0.35]{./images/eeos_2.pdf}\\
{\scriptsize (Margueron, Casali, Gulminelli, in preparation)}
\end{frame}








\section{Observations of NS:}


\begin{frame}{X-ray observations of neutron stars:}
\includegraphics[scale=0.6]{./images/XMM.jpg}\hfill
\includegraphics[scale=0.5]{./images/chandra.jpg}\\
XMM-Newton \hspace{+4cm} Chandra Observatory
\end{frame}


\begin{frame}{Thermal emission of neutron stars:}
\begin{center}
\includegraphics[scale=0.6]{./images/crpnoir2.jpeg}
\end{center}
\end{frame}




\begin{frame}{Observations of thermal emission from NS:}
%$$F \propto T^4\dfrac{R_{\infty}^2}{D^2}$$
black body like emission:
$$F \propto T^4(R_{\infty}/D)^2$$ ({\small Rutledge et al. 1999})
\begin{itemize}
\color{white}
	\item 6 low mass X-ray transcients: almost pure thermal components, low magnetic fields, constant flux, atmosphere composition is purely H, in globular clusters (well constrained distances).
	\item a single EoS $\rightarrow$ simultaneous analysis.
\end{itemize}


\begin{table}
{\small
%\begin{tabular}%{|R{2cm}||L{1cm}|L{2cm}|L{2cm}|L{2cm}|L{2cm}|L{2cm}|}
\begin{tabular}{ccccccc}
\hline Sources & Obs. & distance & nH & atm. & pile-up & Teff \\
  &  & (kpc) & $(10^{22}cm^{-2})$ & & $\alpha$ & (K) \\
\hline M13 & Chandra & $7.1 \pm 0.4$ & $[0;1]$ & H & $[0;1]$ & $[5;6.5]$ \\
 M28 & Chandra & $5.5 \pm 0.3$ & $[0;1]$ & H & $[0;1]$ & $[5;6.5]$ \\
 M30 & Chandra & $9 \pm 0.4$ & $[0;1]$ & H & $[0;1]$ & $[5;6.5]$ \\
 NGC6304 & Chandra & $6.22 \pm 0.26$ & $[0;1]$ & H & $[0;1]$ & $[5;6.5]$ \\
 NGC6397 & Chandra & $2.39 \pm 0.13$ & $[0;1]$ & H & $[0;1]$ & $[5;6.5]$ \\
 $\omega$ Cen & Chandra & $4.59 \pm 0.08$ & $[0;1]$ & H & $[0;1]$ & $[5;6.5]$ \\
\hline
\end{tabular}
}
\end{table}
\end{frame}







\begin{frame}{Modeling the spectra with Xspec:}
\begin{block}{Spectrum model used:}
\begin{itemize}
\item "pile-up" (Davis 2001, Bogdanov 2016), "phabs" absorbtion and "nsatmos" for the atmosphere  (Heinke et al. 2006).
\item $E<2$ keV: only thermal component and no power-law 
\end{itemize}
\end{block}
\begin{block}{Parameters which are allowed to vary:}
\begin{itemize}
\item pile-up alpha
\item hydrogen collumn density on the line of site (nH)
\item distance to the stars (dkpc)
\item the surface effective temperature (Teff)
\item the mass of the stars.
\item the nuclear parameters $K_{sym}$ and $L_{sym}$:
\end{itemize}
\centering
EoS($K_{sym}$,$L_{sym}$,Mass) $\rightarrow$ Radius
\end{block}
\end{frame}








\usebackgroundtemplate{}
\begin{frame}{Reproducing the data:}
MCMC stretch move algorithm with 200 walkers (Mackey et al. 2013)\\
\begin{center}
\includegraphics[scale=0.8]{./images/rchi2_vs_steps_all.pdf}
\end{center}
\end{frame}

\usebackgroundtemplate{\includegraphics[scale=0.8]{./images/backdark.jpg}}

\begin{frame}{Preliminary results (M13 parameters):}
\vspace{-0.5cm}
\begin{center}
\includegraphics[scale=0.17]{./images/corner_d_vary_3000-7350_grp0.png}
\end{center}
\end{frame}

\usebackgroundtemplate{}

\begin{frame}{Conclusion:}
\begin{tabular}{ll}
\hspace{-0.9cm}
\begin{minipage}{0.6\textwidth}
\begin{itemize}

\item Present knowledge of nuclear physics $+$ smooth EoS $\rightarrow R \simeq 11.5-14$ km and $M < 2.5 M_{\odot}$
\item Main source of uncertainties: $L_{sym}$ and $K_{sym}$.
\item if better knowledge on $L_{sym}$ and $K_{sym}$: reduce uncertainty $\simeq 1$ km
\item measurement of R out of these boundaries $\rightarrow$ non-smooth EoS\\
$\rightarrow$ \textbf{Phase transition!} 
\end{itemize}
\end{minipage}
& 
\hspace{-3cm}
\begin{minipage}{0.5\textwidth}
\includegraphics[scale=0.9]{./images/concl.pdf}
\end{minipage}
\end{tabular}
\end{frame}

\usebackgroundtemplate{\includegraphics[scale=0.8]{./images/backdark.jpg}}

\begin{frame}{Outlooks:}
\begin{itemize}
\color{white}
\item Finalizing this study.
\item Extension of present EoS to include \textbf{quark matter.}
\item Further analysis of occurence of a phase transition from X-ray data: \textbf{Bayes factor}.
\item Waiting for $L{sym}$ measurment from PREX.
\item Confronting our results with NICER or NS mergers data. 
\end{itemize}
\end{frame}





\usebackgroundtemplate{\includegraphics[scale = 0.6]{./images/merci}}
\begin{frame}
\begin{center}
\textcolor{white}{ \Large{\textbf{Thank you !}}}
\end{center}
\end{frame}



%BACKUP SLIDES:

\usebackgroundtemplate{}




\end{document}
